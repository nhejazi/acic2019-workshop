\documentclass[]{article}
\usepackage{lmodern}
\usepackage{amssymb,amsmath}
\usepackage{ifxetex,ifluatex}
\usepackage{fixltx2e} % provides \textsubscript
\ifnum 0\ifxetex 1\fi\ifluatex 1\fi=0 % if pdftex
  \usepackage[T1]{fontenc}
  \usepackage[utf8]{inputenc}
\else % if luatex or xelatex
  \ifxetex
    \usepackage{mathspec}
  \else
    \usepackage{fontspec}
  \fi
  \defaultfontfeatures{Ligatures=TeX,Scale=MatchLowercase}
\fi
% use upquote if available, for straight quotes in verbatim environments
\IfFileExists{upquote.sty}{\usepackage{upquote}}{}
% use microtype if available
\IfFileExists{microtype.sty}{%
\usepackage{microtype}
\UseMicrotypeSet[protrusion]{basicmath} % disable protrusion for tt fonts
}{}
\usepackage[margin=1in]{geometry}
\usepackage{hyperref}
\hypersetup{unicode=true,
            pdftitle={Statistical Roadmap and Data Intro},
            pdfauthor={Alan Hubbard},
            pdfborder={0 0 0},
            breaklinks=true}
\urlstyle{same}  % don't use monospace font for urls
\usepackage{graphicx,grffile}
\makeatletter
\def\maxwidth{\ifdim\Gin@nat@width>\linewidth\linewidth\else\Gin@nat@width\fi}
\def\maxheight{\ifdim\Gin@nat@height>\textheight\textheight\else\Gin@nat@height\fi}
\makeatother
% Scale images if necessary, so that they will not overflow the page
% margins by default, and it is still possible to overwrite the defaults
% using explicit options in \includegraphics[width, height, ...]{}
\setkeys{Gin}{width=\maxwidth,height=\maxheight,keepaspectratio}
\IfFileExists{parskip.sty}{%
\usepackage{parskip}
}{% else
\setlength{\parindent}{0pt}
\setlength{\parskip}{6pt plus 2pt minus 1pt}
}
\setlength{\emergencystretch}{3em}  % prevent overfull lines
\providecommand{\tightlist}{%
  \setlength{\itemsep}{0pt}\setlength{\parskip}{0pt}}
\setcounter{secnumdepth}{0}
% Redefines (sub)paragraphs to behave more like sections
\ifx\paragraph\undefined\else
\let\oldparagraph\paragraph
\renewcommand{\paragraph}[1]{\oldparagraph{#1}\mbox{}}
\fi
\ifx\subparagraph\undefined\else
\let\oldsubparagraph\subparagraph
\renewcommand{\subparagraph}[1]{\oldsubparagraph{#1}\mbox{}}
\fi

%%% Use protect on footnotes to avoid problems with footnotes in titles
\let\rmarkdownfootnote\footnote%
\def\footnote{\protect\rmarkdownfootnote}

%%% Change title format to be more compact
\usepackage{titling}

% Create subtitle command for use in maketitle
\providecommand{\subtitle}[1]{
  \posttitle{
    \begin{center}\large#1\end{center}
    }
}

\setlength{\droptitle}{-2em}

  \title{Statistical Roadmap and Data Intro}
    \pretitle{\vspace{\droptitle}\centering\huge}
  \posttitle{\par}
    \author{Alan Hubbard}
    \preauthor{\centering\large\emph}
  \postauthor{\par}
      \predate{\centering\large\emph}
  \postdate{\par}
    \date{2019-03-30}


\begin{document}
\maketitle

\hypertarget{learning-objectives}{%
\subsection{Learning Objectives}\label{learning-objectives}}

\begin{enumerate}
\def\labelenumi{\arabic{enumi}.}
\tightlist
\item
  Introduce the data being used in workshp
\item
  Introduce the statistical roadmap on which the methods are based.
\end{enumerate}

\hypertarget{the-data}{%
\subsection{The Data}\label{the-data}}

The data come from the Effect of water quality, sanitation, hand
washing, and nutritional interventions on child development in rural
Bangladesh (WASH Benefits Bangladesh): a cluster-randomised controlled
trial (Luby et al. (2018)). The study enrolled enrolled pregnant women
in their first or second trimester from rural villages of Gazipur,
Kishoreganj, Mymensingh, and Tangail districts of central Bangladesh,
with an average of eight women per cluster. Groups of eight
geographically adjacent clusters were block-randomised, using a random
number generator, into six intervention groups (all of which received
weekly visits from a community health promoter for the first 6 months
and every 2 weeks for the next 18 months) and a double-sized control
group (no intervention or health promoter visit). The six intervention
groups were: chlorinated drinking water; improved sanitation;
handwashing with soap; combined water, sanitation, and handwashing;
improved nutrition through counselling and provision of lipid-based
nutrient supplements; and combined water, sanitation, handwashing, and
nutrition. In the workshop, we concentrate child-growth (size for age)
as the outcome. This trial was registered with ClinicalTrials.gov,
number NCT01590095.

For the purposes of this workshop, we we start by treating the data as
independent and identically distributed (i.i.d.) random draws from a
very large target population. We could, with available options, account
for the clustering of the data (within sampled geographic units), but
avoiding these details for this workshop.

We have 2 variables and 1 variable set of interest. Our outcome, \(Y\),
is \emph{whz} -- weight For height Z-score, the treatment of interest,
\(A\), is \emph{tr} -- randomized treatment group, and our adjustment
variables, \(W\), are \emph{everything else}. This results in our
observed data structure as \(i=1,...,n\) i.i.d. copies of
\(O_i = (W_i,A_i,Y_i)\).

Modifications of our methods for biased samples, repeated measures, etc.
are available.

\hypertarget{the-variables}{%
\subsubsection{The variables}\label{the-variables}}

% latex table generated in R 3.5.3 by xtable 1.8-3 package
% Sat Mar 30 13:38:11 2019
\begin{table}[ht]
\centering
\caption{Distribution of variables used in WASH-Benefits Study} 
\label{table_1}
\begin{tabular}{|l|r|c|c||c|c|}
  \hline
Variable & levels & Mean & SD & Count & Row.Prop \\ 
  \hline
whz &  & -0.5861 & 1.0321 &  &  \\ 
   \hline
month &  & 6.4547 & 3.3321 &  &  \\ 
   \hline
aged &  & 266.3150 & 52.1746 &  &  \\ 
   \hline
momage &  & 23.9059 & 5.2405 &  &  \\ 
   \hline
momheight &  & 150.5041 & 5.2267 &  &  \\ 
   \hline
Nlt18 &  & 1.6047 & 1.2473 &  &  \\ 
   \hline
Ncomp &  & 11.0432 & 6.3504 &  &  \\ 
   \hline
watmin &  & 0.9487 & 9.4812 &  &  \\ 
   \hline
\hline
tr & Control &  &  &  1178 & 0.2509 \\ 
   & Handwashing &  &  &   590 & 0.1257 \\ 
   & Nutrition &  &  &   570 & 0.1214 \\ 
   & Nutrition + WSH &  &  &   598 & 0.1274 \\ 
   & Sanitation &  &  &   590 & 0.1257 \\ 
   & Water &  &  &   582 & 0.1240 \\ 
   & WSH &  &  &   587 & 0.1250 \\ 
   \hline
sex & female &  &  &  2352 & 0.5010 \\ 
   & male &  &  &  2343 & 0.4990 \\ 
   \hline
momedu & No education &  &  &   734 & 0.1563 \\ 
   & Primary (1-5y) &  &  &  1448 & 0.3084 \\ 
   & Secondary ($>$5y) &  &  &  2513 & 0.5353 \\ 
   \hline
hfiacat & Food Secure &  &  &  3249 & 0.6920 \\ 
   & Mildly Food Insecure &  &  &   410 & 0.0873 \\ 
   & Moderately Food Insecure &  &  &   879 & 0.1872 \\ 
   & Severely Food Insecure &  &  &   157 & 0.0334 \\ 
   \hline
elec & 0 &  &  &  1901 & 0.4049 \\ 
   & 1 &  &  &  2794 & 0.5951 \\ 
   \hline
floor & 0 &  &  &  4194 & 0.8933 \\ 
   & 1 &  &  &   501 & 0.1067 \\ 
   \hline
walls & 0 &  &  &  1338 & 0.2850 \\ 
   & 1 &  &  &  3357 & 0.7150 \\ 
   \hline
roof & 0 &  &  &    69 & 0.0147 \\ 
   & 1 &  &  &  4626 & 0.9853 \\ 
   \hline
asset\_wardrobe & 0 &  &  &  3910 & 0.8328 \\ 
   & 1 &  &  &   785 & 0.1672 \\ 
   \hline
asset\_table & 0 &  &  &  1247 & 0.2656 \\ 
   & 1 &  &  &  3448 & 0.7344 \\ 
   \hline
asset\_chair & 0 &  &  &  1247 & 0.2656 \\ 
   & 1 &  &  &  3448 & 0.7344 \\ 
   \hline
asset\_khat & 0 &  &  &  1816 & 0.3868 \\ 
   & 1 &  &  &  2879 & 0.6132 \\ 
   \hline
asset\_chouki & 0 &  &  &  1027 & 0.2187 \\ 
   & 1 &  &  &  3668 & 0.7813 \\ 
   \hline
asset\_tv & 0 &  &  &  3268 & 0.6961 \\ 
   & 1 &  &  &  1427 & 0.3039 \\ 
   \hline
asset\_refrig & 0 &  &  &  4322 & 0.9206 \\ 
   & 1 &  &  &   373 & 0.0794 \\ 
   \hline
asset\_bike & 0 &  &  &  3197 & 0.6809 \\ 
   & 1 &  &  &  1498 & 0.3191 \\ 
   \hline
asset\_moto & 0 &  &  &  4385 & 0.9340 \\ 
   & 1 &  &  &   310 & 0.0660 \\ 
   \hline
asset\_sewmach & 0 &  &  &  4391 & 0.9353 \\ 
   & 1 &  &  &   304 & 0.0647 \\ 
   \hline
asset\_mobile & 0 &  &  &   664 & 0.1414 \\ 
   & 1 &  &  &  4031 & 0.8586 \\ 
  \end{tabular}
\end{table}


The distribution of relevant variables is shown in Table \ref{table_1}.
The top rows summarize continuous variables, the remainder are
categorical. The asset variables reflect the socio-economic status of
the subjects. Notice also the uniform distribution of the treatment
groups (with twice as many controls); this is of course by design.

\hypertarget{the-model}{%
\subsection{The model}\label{the-model}}

The reason you're taking this course, presumably, is because you want to
estimate relevant parameters in realistic models.

\hypertarget{the-statistical-model}{%
\subsubsection{The statistical Model}\label{the-statistical-model}}

One can break up the distribution of the observed data as follows:
\(P(O)=P(W,A,Y)=P(W)P(A|W)P(Y|A,W)\). To estimate a parameter of
interest, the researcher does not necessarily need to specify these
whole distributions and conditional distributions. Each estimator
requires certain parts of the distribution, for example some require
estimates of \(E(Y|A,W)=\) mean of \(Y\) within subgroups \((A,W)\), or
the regression of the outcome on the exposure and confounders. At this
stage in the road map the researcher specifies what statistical model
she will use to estimate \(E(Y|A,W)\) or other elements of the
probability distribution that are needed to estimate the parameter of
interest. By ``statistical model'' here we mean any constraints on the
model form that are imposed by knowledge about the process - known
aspects of how the data were generated. Typically, the true model is a
very big model, with few constraints if any on the data-generating
distribution, or a semi-parametric model. Thus, with few constraints on
the data-generating distribution, and a potentially large number of
covariates, data-adaptive, machine-learning approaches are the only
practical option. The remainder of the course is how to do this as
efficiently and robustly as possible, depending on the goal of the
analysis.

\hypertarget{the-causal-model}{%
\subsubsection{The Causal Model}\label{the-causal-model}}

The next step is to use a causal framework to formalize the experiment
and thereby define the parameter of interest. Causal graphs are one
useful tool to express what we know about the causal relations among
variables that are relevant to the question under study (Pearl (2000)).
An illustration shows a simple causal graph, specifically a directed
acyclic graph or DAG, to depict the causal relations between variables
with a simple example of a binary exposure A, a binary outcome Y, and
one categorical confounding variable W. The DAG for these relations is
depicted below.

\includegraphics[width=4.0in]{CausalGraph.pdf}

The \(U_W\), \(U_A\), and \(U_Y\) are the unmeasured exogenous
background characteristics that influence the value of each variable.
Alternatively, the same causal relations among variables can be
represented with a series of equations: \begin{eqnarray*}
W &=& f_W(U_W) \\
A &=& f_A(W,U_A) \\
Y&=&f_Y(W,A,U_Y). 
\end{eqnarray*}

Here \(f_W, f_A\) and \(f_Y\) denote that each variable (\(W, A\) and
\(Y\) respectively) is a function of its parents and unmeasured
background characteristics, but there is no imposition of any particular
functional constraints. For this reason, they are called non-parametric
structural equations. The DAG and this series of non-parametric
structural equations represent the same information.

\hypertarget{parameter-of-interest}{%
\subsection{Parameter of Interest}\label{parameter-of-interest}}

The first hypothetical experiment we will consider is assigning exposure
to the whole population and observing the outcome, and then assigning no
exposure to the whole population and observing the outcome. On the
non-parametric structural equations, this corresponds to a comparison of
the outcome distribution in the population under two interventions: 1) A
is set to 1 for all individuals, and 2) A is set to 0 for all
individuals. These interventions imply two new non-parametric structural
equation models with first being: \begin{eqnarray*}
W &=& f_W(U_W) \\
A &=& 1 \\
Y(1)&=&f_Y(W,1,U_Y), 
\end{eqnarray*} and second just replacing the intervention of \(A=1\) to
\(A=0\), \begin{eqnarray*}
W &=& f_W(U_W) \\
A &=& 0 \\
Y(0)&=&f_Y(W,0,U_Y).
\end{eqnarray*}

In these equations, A is no longer a function of W because we have
intervened on the graph and set A to the values 1 and 0. The new symbols
Y(1) and Y(0) indicate the outcome variable in our population if it were
generated by the respective NPSEMs above; these are often called
counterfactuals. The difference between the means of the outcome under
these two interventions defines a parameter that is often called the
``Average Treatment Effect (ATE), or \begin{equation}
\label{ate}
ATE = E_X(Y(1)-Y(0)),
\end{equation} \#\#\# Identifiability where \(E_X\) is the mean under
the theoretical full data: \(X=(W,Y(1),Y(0))\). Because we can never
observe both \(Y(0)\) (counterfactual when \(A=0\)) and \(Y(1)\), we can
not estimate \ref{ate} directly. Thus, we have to make assumptions to
estimate this quantity from \(O \sim P_0\), or the data-generating
distribution. Fortunately, given our causal model shown in the graph
above, we can, with a couple of more assumptions, estimate the ATE even
from observational data. First, the causal graph implies that
\(Y(a) \perp A\) for all \(a \in \mathcal{A}\), which is the
randomization assumption. Outside of the graph, one also needs a
non-interference assumption (say the \(Y\)'s are independent) and
positivity (\(0<P_0(A=a\mid W)< 1\) for all \(a\) and \(W\)). Given
these assumptions, then one can re-write the ATE as a function of
\(P_0\), specifically \begin{equation}
\label{estimand}
ATE = E_0(Y(1)=Y(0)) = E_0\left( E_0[Y \mid A=1,W]-E_0[Y \mid A=0,W] \right),
\end{equation} or the difference in predicted values for each subject in
population and then averaging over those subjects. Thus, a parameter of
a theoretical''full" data distribution can be represented as an estimand
of the observed data-distribution. Significantly, there is nothing about
the representation in \ref{estimand} that requires parameteric
assumptions, so that the regressions on the right hand side can be
estimated freely with machine learning. With different parameters, there
will be potentially different identifiability assumptions and the
resulting estimands can be functions of different components of \(P_0\).
We discuss several more complex ones in this workshop later.

\hypertarget{estimator}{%
\subsection{Estimator}\label{estimator}}

We will discuss more in the specific sections later, but the goals of
the estimator should be, among sensible (asymptotically consistent,
regular) estimators 1. They are asymptotically efficient in the
statistical model of interest. 2. They can be constructed for finite
sample performance improvements.

\hypertarget{inference}{%
\subsection{Inference}\label{inference}}

The estimators we discuss are asymptotically linear, meaning that the
difference in teh estimate \(\Psi(P_n)\) and the true parameter
(\(\Psi(P_0)\)) can be represented in first order by a i.i.d. sum:
\begin{equation}
\label{IC}
\Psi(P_n) - \Psi(P_0) = \frac{1}{n} IC(O_i; \nu)+op(1/\sqrt{n})
\end{equation} where \(IC(O_i; \nu)\) is a function of the data and
possibly other parameters, \(\nu\). Thus, in the univariate case, one
can derive a 95\% confidence interval as: \begin{equation}
\label{CI}
\Psi(P^*_n) \pm 1.96 \sqrt{\frac{\hat{\sigma}^2}{n}}
\end{equation}

where \(SE=\sqrt{\frac{\hat{\sigma}^2}{n}}\) and \(\hat{\sigma}^2\) is
the sample variance of the estimated IC's: \(IC(O;\hat{\nu})\). One can
use the functional delta method to derive the influence curve if a
parameter of interest that is written as a function of other
asymptotically linear estimators.

\hypertarget{references}{%
\subsection*{References}\label{references}}
\addcontentsline{toc}{subsection}{References}

\hypertarget{refs}{}
\leavevmode\hypertarget{ref-luby2018effects}{}%
Luby, Stephen P, Mahbubur Rahman, Benjamin F Arnold, Leanne Unicomb,
Sania Ashraf, Peter J Winch, Christine P Stewart, et al. 2018. ``Effects
of Water Quality, Sanitation, Handwashing, and Nutritional Interventions
on Diarrhoea and Child Growth in Rural Bangladesh: A Cluster Randomised
Controlled Trial.'' \emph{The Lancet Global Health} 6 (3). Elsevier:
e302--e315.

\leavevmode\hypertarget{ref-pearl_causality_2000}{}%
Pearl, J. 2000. \emph{Causality}. Cambridge University Press.


\end{document}
